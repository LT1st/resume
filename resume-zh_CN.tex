% !TEX TS-program = xelatex
% !TEX encoding = UTF-8 Unicode
% !Mode:: "TeX:UTF-8"

\documentclass{resume}
\usepackage{zh_CN-Adobefonts_external} % Simplified Chinese Support using external fonts (./fonts/zh_CN-Adobe/)
%\usepackage{zh_CN-Adobefonts_internal} % Simplified Chinese Support using system fonts
\usepackage{linespacing_fix} % disable extra space before next section
\usepackage{cite}
\usepackage{hyperref}
\hypersetup{hidelinks,
	colorlinks=true,
	allcolors=black,
	pdfstartview=Fit,
	breaklinks=true}


\begin{document}
\pagenumbering{gobble} % suppress displaying page number

\name{卢韬}

% {E-mail}{mobilephone}{homepage}
% be careful of _ in emaill address
\contactInfo{188-0403-0926}{1417274896@qq.com}{计算机视觉方向}{GitHub \href{https://github.com/LT1st}{@lt1st}}
% {E-mail}{mobilephone}
% keep the last empty braces!
%\contactInfo{xxx@yuanbin.me}{(+86) 131-221-87xxx}{}
 
\section{个人介绍}
本人乐观向上,工作负责、自我驱动力强、热爱尝试新事物、抗压能力强、身体素质好。
在校期间长期从事计算机视觉相关研究,对图像处理、深度学习有较深理解。
熟悉图像分类、目标检测、语义分割等视觉领域常见Baseline,有快速复现论文的能力。
阅读西瓜书、统计学习方法、数据挖掘等多部数理统计书籍,对算法数学原理有一定了解。


% \section{\faGraduationCap\ 教育背景}
\section{教育背景}
\datedsubsection{\textbf{东北大学},信息科学与工程学院,自动化专业,\textit{本科在读}}{2019.9 - 2023.6}
%\ \textbf{排名150/260(前60\%)}

% \section{\faCogs\ IT 技能}
\section{技术能力}
% increase linespacing [parsep=0.5ex]
\begin{itemize}[parsep=0.2ex]
  \item \textbf{编程语言}: Python, C(STM32) , C++, Matlab, R, Verilog, JAVA (Tomcat,Servelet), HTML
  \item \textbf{IT技能}: Git, Shell , Linux, Markdown, LaTex, MySQL, 良好的代码风格
  \item \textbf{英语能力}: CET-6(528), 学术英语写作(97), 英语翻译(95), 大学英语(90)
  \item \textbf{计算机视觉}: OpenCV(C++,Python), Pytorch , PIL , PCL(点云) 
  \item \textbf{关键词}: ReID , Object detection(Small Obstacl) , Instance Segmentation , Measurement
\end{itemize}

% \end{itemize}

\section{科研经历}
\datedsubsection{\textbf{海康威视研究院 | Hikvision Research Institute}, 视觉算法工程师}{2021.12 - 2022.03}
\begin{itemize}
  \item 分析现有算法在自动驾驶障碍物检测中的缺陷,利用可视化找出低矮、凹陷、细薄悬空等难样本。通过阅读\textbf{六十余篇}论文进行文献综述,针对难样本形成4套可靠的检测方案,输出研究报告2份。   
  \item 参与会议讨论,共同制定障碍物检测方案,并根据讨论\textbf{从零完成论文复现}(\href{https://github.com/LT1st/SmallObstacleDetection}{Small Obstacl Detection})。
\end{itemize}

\datedsubsection{\textbf{智能工业数据解析与优化教育部重点实验室 | DAO},优化算法工程师}{2021.06 - 至今}
\begin{itemize}
  \item 应用\textbf{智能优化算法}到晶胞参数优化中。基于matlab和c++实现遗传、差分进化、粒子群优化算法。
  \item 差分进化算法优化BP神经网络初始权重和超参数,使收敛epoch\textbf{从200降到14},效果显著。
  \item 利用遗传算法、粒子群优化算法求解指派问题、TSP问题,同时针对性优化算法各个环节。
\end{itemize}


\datedsubsection{\textbf{创新机器人实验室 | Action},视觉算法工程师}{2020.05 - 2020.12}
\begin{itemize}
  \item 负责机器人感知,与控制、机械设计确定感知策略、数据传输协议、数据结构等。
  \item 完成单片机与电脑的通讯方案,在stm32和linux环境自定义了蓝牙、串口、CAN传输方案。 
  \item 使用realsense相机,用OpenCV、CUDA和PCL在边缘设备TX2进行定位、姿态感知、物体识别。
\end{itemize}

% \begin{onehalfspacing}
% \end{onehalfspacing}

\section{项目经历}
\datedsubsection{\textbf{第十四批大创项目 | 基于c++的射影定理可视化平台},预备成员}{2020.08 - 2022.03}
\begin{itemize}
  \item 根据理论推导结果,使用轻量化的easyx库(c++)实现射影定理算法可视化开发。
\end{itemize}
\datedsubsection{\textbf{第十五批大创项目 | 基于视觉重识别的远距离静默考勤系统},核心成员}{2019.09 - 2020.08}
\begin{itemize}
  \item 调研已有Reid算法,根据论文读懂对应开源代码,做详细注释,为后续改进工作打下基础。
  \item 搭建实验平台,实现了基于Pytorch的Reid和online learning算法。 
  \item 项目获评省级大创项目、我最喜爱的项目,结题表现为良好。
\end{itemize}


\section{竞赛获奖}
% increase linespacing [parsep=0.5ex]
\begin{itemize}[parsep=0.2ex]
  \item 目标检测比赛(Starfish Detection), 铜牌(193/2026),2021
  \\ \qquad 使用:YOLOX,YOLOV5,YOLOR,FasterRCNN,时序交叉模型融合,手写NMS 等
  \item 美国数学建模竞赛,三等奖 , 2021,2022
  \\ \qquad 使用:决策树(CRT、XGBOOST),LSTM,DQL等
\end{itemize}

% \section{\faHeartO\ 项目/作品摘要}
% \section{项目/作品摘要}
% \datedline{\textit{An Integrated Version of Security Monitor Vis System}, https://hijiangtao.github.io/ss-vis-component/ }{}
% \datedline{\textit{Dark-Tech}, https://github.com/hijiangtao/dark-tech/ }{}
% \datedline{\textit{融合社交网络数据挖掘的电视节目可视化分析系统}, https://hijiangtao.github.io/variety-show-hot-spot-vis/}{}
% \datedline{\textit{LeetCodeOJ Solutions}, https://github.com/hijiangtao/LeetCodeOJ}{}
% \datedline{\textit{Info-Vis}, https://github.com/ISCAS-VIS/infovis-ucas}{}


% \section{\faInfo\ 社会实践/其他}
\section{社区参与/实践其他}
% increase linespacing [parsep=0.5ex]
\begin{itemize}[parsep=0.2ex]
  \item 乐于参与开源社区讨论,为多个项目提issue,修改BUG。\href{https://blog.csdn.net/prinTao?type=blog}{技术博客阅读量\textbf{3.1W}}。
  \item 校党委宣传部记者,学院团委宣传部记者,校科协实习副部长,长跑队、游泳队成员。
  \item CCC2020优秀志愿者,ICPC2021优秀志愿者,忻州市统计局计算站实习生,总社会服务时间\textbf{200h+}
\end{itemize}

%% Reference
%\newpage
%\bibliographystyle{IEEETran}
%\bibliography{mycite}
\end{document}
